\section{Evaluation}\label{sec:dspmv-evaluation}
%-our program behaves the "same" as the Bylina et al paper. 


\subsection{Cluster Environment and Methodology}
%- cluster architecture, network connectivity, number of nodes (max used), etc.

For our tests we utilized a 64 node computing cluster utilizing IBM NeXtScale nx360 M4 servers each with dual 8-core Intel(R) Xeon(R) CPU E5-2650 v2 at 2.60GHz. 
The hardware specifications for each server provided two sockets and 16cores per node, for a total of 1024 cores available within the cluster. 
Finally, cluster nodes where connected via Mellanox FDR non-blocking Infiniband.

Our version of the sub matrix method was compiled on the test environment using the mpic++ compiler for OpenMPI 2.0.1, with pass through to gcc/g++ 6.2.0 and OpenMP version 4.5. 
When compiling, the optimization level 3 flag to allow gcc to utilize all default optimization schemes for that optimization level. 

Time measurements taken during tests where performed using the MPI\_Wtime() method, with nanosecond clock precision, in all MPI process and OpenMP thread count per process variations examined examined. 
Three distinct time measurements were taken for each test run: distribution determination, SpMV computation, and total elapsed time. 
While the distribution determination timings will provide valuable insight in our future works, we will focus on the those timings recorded for the computation of SpMV across all MPI processes.
SpMV computation times where taken after each process had received its work allotment, and again after all nodes had completed their computation. 
This was done to insure that time measurements incorporate differences in runtime between processes created by the potential for workload imbalances inherent to the sub matrix distribution method.

\begin{table}[]
	\centering
	\caption{Properties of Benchmark Matrices}
	\label{matrix-properties}
	\begin{tabular}{lrrl}
		\hline
		\multicolumn{1}{|c|}{\textbf{Name}} & \multicolumn{1}{c|}{\textbf{Rows/Cols}} & \multicolumn{1}{c|}{\textbf{Non-Zeros}} & \multicolumn{1}{l|}{\textbf{Density}} \\ \hline
		parabolic\_fem                      & 525825                             & 3674625                                 & 6.98                                   \\
		bmw3\_2                             & 227632                             & 5757996                                 & 49.59                                  \\
		torso1                              & 1116158                            & 8516500                                 & 73.32                                  \\
		nd24k                               & 72000                              & 28715634                                & 398.83                            
	\end{tabular}
\end{table}

During our tests, we used the same matrices as analyzed in \cite{techbib:6933066}. Table \ref{matrix-properties} lists the properties of each matrix including the number of rows/cols, total number of non-zeros, as well as the matrix's density. Density for this purpose is calculated as $rows / non-zeros$. Additionally all matrices being evaluated are square $n$x$n$ symmetric matrices. 


%- number of tests run per test permuation
%	- best, worst, and averages were taken for all times recorded and GFlops calculated
%	
%- then tables and pretty graphs to talk about 
%
%
%- As we add threads, we are adding memory access (bandwidth) due to the usage or more cores on a chip
%
%- as we increase MPI processes we increase the number of processors, and threfore the potential for more threads.
%
%-performance as a function of the number of non-zeros per row. What I mean by this is that as you increase the number of processes, you split up rows into smaller sections, thereby decreasing the of non-zero elements in a %given row for each submatrix. 
%	\-eventually there will be so few non-zeros that you will have nothing to do and only overhead to keep you busy (performance will drop).
%	
%Trade off space:
%	- \# of processes
%	- threads per process
%	- \# of processes per socket
%	- \# of nodes utilized
%	- matrix size
%	- \# of non-zeros